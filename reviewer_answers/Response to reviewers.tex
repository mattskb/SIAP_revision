\documentclass[a4paper,11pt]{scrartcl}
%\usepackage[utf8]{inputenc}
\usepackage{amssymb,amsmath,amsthm}
\usepackage{array, hhline}
\usepackage{tikz}
\usepackage{authblk}


\usepackage[]{times}
\usepackage{fullpage}
%\addtolength{\textheight}{1cm}
\usepackage{graphicx}
%\usepackage{subfigure}

\date{}%leave empty

%added by Florin

\usepackage{color}
\newcommand{\red}{\textcolor{red}}

%\renewcommand{\topmargin}{-36pt}
%\renewcommand{\headheight}{36pt}


%\newcommand{\beginproof}{{\bf Proof}}
%\newcommand{\finishedproof}{\hfill {\bf Q. E. D.}}
%
%\newcommand{\gr}[1]{\ensuremath{\nabla_{\bf #1}}} %gradient wrt a vector
%\newcommand{\ve}[1]{\ensuremath{\bf{ #1}}}				%bold vectors in math mode
%\newcommand{\bg}[1]{\ensuremath{\boldsymbol{#1}}} %bold greek symbols
%
%\newcommand{\ds}{\displaystyle}
%\newcommand{\R}{\mathbb{R}}
%\newcommand{\N}{\mathbb{N}}
%
%
%
%\def\l2{L^2(\Omega)}
%\def\ha1{H^1(\Omega)}
%\def\h10{H^1_0(\Omega)}
%\def\hdiv{H(\mathrm{div}; \Omega)}

%\def\R{{\rm I\kern-.17em R}}
%\define\ker
%sonstige
%\def\o{\Omega}
%\def\into{\int_\Omega}
%\def\normala{\overline{n}}
%\def\oq{\overline{{\bf q}}}
%\def\oc{\overline{c}}
%\def\ocQ{\overline{c {\bf Q} }}
%\def\oQ{\overline{ {\bf Q} }}
%\def\bun{\overline{{\bf u}}^n}
%\def\osn{\overline{{s}}^n}
%\def\opn{\overline{{p}}^n}
%\def\othetan{\overline{\Theta}^n}
%\def\oqn{\overline{{\bf q}}^n}
%\def\oqk{\overline{{\bf q}}^k}
%
%\def\bq{{\bf q}}
%\def\bv{{\bf v}}
%\def\bQ{{\bf Q}}
%\def\be{{\bf e}}
%\def\bx{{\bf x}}
%\def\bu{{\bf u}}
%\def\bff{{\bf f}}
%\def\ov{\overline{{\bf v}}}
%\newcommand{\ds}{\displaystyle}
%\newcommand{\dfrac}[2]{\frac{ \ds #1}{\ds #2}}
%\newcommand{\ra}{\rangle}
%\newcommand{\la}{\langle}
%\newtheorem{problem}{\sc \bf Problem}[section]
%\newtheorem{proposition}{\sc \bf Proposition}[section]
%\newtheorem{remark}{\bf Remark}[section]
%\newtheorem{theorem}{\bf Theorem}[section]
%\newtheorem{lemma}{\bf Lemma}[section]

\newcommand{\Wcal}{\mathcal{W}}
\newcommand{\Ucal}{\mathcal{U}}
\newcommand{\Pcal}{\mathcal{P}}
\newcommand{\Tcal}{\mathcal{T}}
\newcommand{\Rcal}{\mathcal{R}}
\newcommand{\Acal}{\mathcal{A}}
\newcommand{\Scal}{\mathcal{S}}
\newcommand{\Bcal}{\mathcal{B}}
\newcommand{\Ccal}{\mathcal{C}}
\newcommand{\Xcal}{\mathcal{X}}
\newcommand{\Jcal}{\mathcal{J}}
\newcommand{\Mcal}{\mathcal{M}}
\newcommand{\Hcal}{\mathcal{H}}

\let\phi\varphi
\begin{document}

\textbf{Dear editorial office,}\\

Please, find attached the revised version of our manuscript \#M140522: "Modeling the process of speciation using a multi-scale framework including a posteriori error estimates". In this revision we have addressed all issues raised by the referees, to whom we are truly thankful for the thorough reading and for the suggestions that helped us in improving the paper. The changes made to the manuscript are colored red for easier comparison with the original version. You can find below detailed answers at reviewers comments and the modifications made to the manuscript. Many thanks,\\

Mats K. Brun, Elyes Ahmed, Jan M. Nordbotten and Nils Chr. Stenseth.\\[2ex]

\textbf{ Referee \#1  (Remarks to the Author)}\\

The paper deals with the process of speciation, which is an inportant process within evolutionary biology. The emphasis of the paper is on the posterior error estimation in a multi-scale framework. The paper is well-written and well-organised. The mathematical theories that have been developed are sound. I do not necessarily agree with the modelling framework, but this is not what the current manuscript is about, hence this remark of mine is irrelevant. I recommend minor revisions, and to aid the Authors present their work more clearly, I have the following points that are to be addressed:

\begin{enumerate}
\item
In general, the abstract may put more emphasis on what the paper is actually about: namely the posterior error analysis. This has only been written in the very last sentences of the abstract. Probably, this is a matter of taste ... \\

AUTHOR answer: We agree with this point. The entire abstract has been re-written to better reflect the actual contents of the paper.

\item
The diffusion (second spatial derivative) term in the left-hand side of equation (2.2) represents the evolutionary process. I would call g the diffusion tensor instead of the diffusion coefficient (line 180).

AUTHOR answer: This has been corrected. We searched for other instances of "diffusion coefficient" aswell, but there were none.

\item
Since g is the diffusion tensor. It is assumed to be symmetric (Assumption 3.1). Of course, this symmetry facilitates the mathematical treatment with respect to well-posedness and error estimation. It is said that in the manuscript that Assumptions 3.1 follow from biology. Could there be any biological application cases where g is nonsymmetric? \\

AUTHOR answer: 
%Jeg tror ikke vi trenger symmetri, vi trenger sannsynligvis bare at den er positiv definit. Symmetri gir reelle egenverdier, og jeg har alltid tenkt at det var naturlig. Men n� som jeg reflekterer over det er det kanskje mulig (alt er mulig i biologi) med komplekse egenverdier, s� lenge den reelle parten er positiv. 

\item
Regarding the boundary condition that n vanishes on the boundary of the trait space (assumung that the trait space is 'large enough'), one could also relax this BC by considering g grad n + A n = 0. Although this BC would relax that n = 0, one has to fiddle around with the A-parameter. Do you think that this alternative BC could do a better job?

AUTHOR answer:
%For finite domains, it is true that this is possible. I think it suffices to answer that we agree with the reviewer that this is conceptually an option, but we prefer to avoid introducing more (spatially dependent) parameter functions, and therefore retain the exposition with zero BCs. 

\item
Regarding the modelling part. The g-tensor for diffusion is based on 'ordinary' diffusion in which random changes are gradual. In many biological systems, mutations could give large, sudden changes over time, which could be modelled using Levy-jump processes. Then one has to incorporate fractional derivatives. Perhaps the Authors could comment on this. (I am not asking the Authors to carry out the analysis for this case, that would really be too much, the question is just meant to be philosophical).

AUTHOR answer: Yes, we can include a comment on fractional derivatives. \red{TODO}

\item
The multi-species model models the transition between species, and in a short subsection (2.3), the relation between the multi-species model and the polultion level model is presented. What I miss om the multi-species model is the actual evolutionary part of the process. There is no random part in this model, which does exist in the population level model through the volatility (diffusion) part. Should not the Authors include a Wiener process (or Levy process to account for jump processes) here (to allow random walk)? Now the approach seems entirely deterministic.

AUTHOR answer: The diffusion (evolutionary part) enters the species-level model through equations (2.5c) and (2.6d), i.e., $V_{0,i} := 2g_i$.

\item
Section 3.3.4, regarding uniqueness. I think that the Authors should add that relation (3.7c) (monotone) makes $||R^k||$ convex, which then proves uniqueness. Formally, the Authors only say that $\phi = n^k$ makes $||R^k||$ zero (which is the minimum), but this formally does not yet exclude any other possible minima.

AUTHOR answer: Indeed, thanks for pointing this out to us. Section 3.3.4 has now been updated according to the above comment.

\item
Line 528 please, replace 'The figure 2 below' with 'Figure 2 below'

AUTHOR answer: This has been corrected.

\item
Your analysis is also built on the existence of statistical moments, hence in the case of a Cauchy distribution (which has a very 'normal' appearance), the model may lead to problems. It could be interesting to do a numerical experiment with a Cauchy distribution function. ($f(x) = 1/\pi \cdot 1 / (x^2 + 1)$ is the standard Cauchy distribution, which has no moments (average, standard deviation ...).

AUTHOR answer: The derivation of the species level model from the population level model rests on two fundamental assumptions; (1) that distinct species can be identified at all times over the global population distribution (i.e., as compactly supported, non-overlapping sub-distributions), and (2) that these sub-distributions are of a known statistical quality (in this case, the Gaussian/normal distribution). As discussed in [1] (where the species level model is derived), the choice of the Gaussian was to a certain degree due to convenience (although individual traits are commonly assumed to follow a normal distribution in the biological literature), hence other statistical distributions could also be accomodated (at the cost of an increasingly complex calculation/notation, see e.g., page 5-6 in [1] where the possibility of using the uniform distribution is briefly discussed). However, for the exact reasons given in your comment, the Cauchy distribution is not a possible choice, although Cauchy distributions do not pose a problem in terms of the population level model. \\

In the manuscript we emphasized the fact that speciation events will invalidate the species level model (violation of assumption (1)), but it may also be invalidated by the species distributions being of an unknown quality (or, with non-existing moments, e.g., Cauchy), hence violating assumption (2). It appears we have neglected to mention this. In the fifth paragraph of the introduction (pages 2-3), where we discuss the species and population level models, we have added a few sentences outlining the above point concerning assumption (2). We have also added in section (2.3) that here $n$ should be smooth enough for these moments to exist.\\

%These assumptions may be quite restrictive when modeling an eco-evolutionary system, but such is the cost when averaging the global trait-dependent population distribution/density into a state vector (perhaps coupling the population and species levels in a multi-scale framework might be worth a try...).  \\


     

\item
Proof of Proposition 6.1: I would replace 'By construction, we have' with 'By construction, from relations (6.2) and (6.3), we have'

AUTHOR answer: Indeed, this has been changed.

\item
line 617: please, replace 'leads' with 'lead'

AUTHOR answer: This has been corrected.

\item
line 664: The Authors note that, indeed, the Gaussian does not satisfy the boundary conditions. Indeed, if the boundaries are far enough, then, the boundary conditions are approximately satisfied. Hence, then there is no problem. The Authors could use the classical bound for a normal distribution (which is much sharper than Chebychev's Inequality) that

$$P(|X| \ge t ) \le \sqrt{\frac{2}{\pi}} \frac{\sigma}{t} exp(-\frac{t^2}{2 \sigma^2}),$$

where $X \sim \mathcal{N}(0,\sigma^2)$. Of course, this relation is based on independence. Using the eigenvalues of the covariance (which is symmetric and hence the eigenvectors are orthogonal), the expression could nevertheless provide some estimation of the upper bound of the initial condition on the boundary.
Of course, this is not necessary to quantify directly for this manuscript, but the Authors could incorporate this in their future studies on this matter.

AUTHOR answer: We agree this is not necessary to quantify for the present purposes, but in any case thank you for pointing this out to us, it may very well become useful for future work on this topic.

\item
The errors in the figures, as well as the tolerances used in the paper (for instance $TOL_{res}$) seem to be large to me. Perhaps, they are not large relatively. The Authors should put this into its context, I think.

AUTHOR answer: We chose not to normalize the errors and estimators used in the numerics such that the a posteriori modeling-remainder estimator, $\eta_{rem,i}^k$ (shown in figures 6 and 13) is calculated exactly as given by Theorem 6.3, eq. (6.12d). The reason for such large values is that $\eta_{rem,i}^k$ is scaled by $G^{-\frac{1}{2}}$, where $G$ is the largest eigenvalue of the diffusion tensor $g$, which is chosen as a rather small constant for both examples (i.e., $5e-6$ and $2e-6$, respectively). We have added a sentence in Section 7, third paragraph where we comment on this.

\item
Line 768: Please specify which figures you are describing in terms of their labels rather than talking about 'The figures below'.

AUTHOR answer: This paragraph has been rewritten to make it more clear which figures are referred to. Similarly with the first paragraph of section (7.2.2).

\item
Please, say something about a comparison of computing times of the heuristic method and the multiscale method in Sectino 7.3.2.

AUTHOR answer: A new remark has been added on p. 23 (remark (7.1)) where we address computation times of the heuristic and multi-scale methods.

\item
Figures 11 and 12 seem blurry (or is it my eyes after having read the manuscript).

AUTHOR answer: These figures do not appear blurry to the author (who possess eagle eyed vision, of course). Source files have been double checked, everything should be in order.

\item
These were my comments. I appreciate the paper, I enjoyed reading.

AUTHOR answer: Thank you for the positive evaluation of our manuscrupt, we really appreciated your thoughtful comments!
\end{enumerate}
%%%%%%%%%%%%%%%%%%%%%%%%%%%%%%%
\textbf{Referee \#2  (Remarks to the Author):}\\

This manuscript is concerned with a mathematical modelling for speciation using a general Lotka-Volterra equation modelling the evolution of some ecosystem. This model takes into account species traits and mutations through a diffusion term. This general equation is coupled with a system of ODE modelling the species level: density of species together with the evolution of the mean traits and the covariance matrix.

\begin{enumerate}
\item
The authors discuss and propose a model for the speciation processes leading to new species. To do so, they make use of a discretized version of the above equations. The discussions on the speciation processes seems to be highly related to the choice of the discrete system and in particular on the reconstruction operator. No connection with the original continuous system is clearly explained neither how it depends on the discretization.
Speciation is modeled in term of clustering in the trait space. Does it depend on the traits kept in the model?

AUTHOR answer: 

%Oi. Denne revieweren har ikke forst�tt noenting. Jeg foresl�r at vi h�fflig svarer at vi bruker konvergent diskretiseringsmetode, og at kvaliteten p� rekonstruksjonsoperatoren styres av a posteriori feilestimatene. 

\item
From a modelling view point, how do you choose the selection function $\alpha=\alpha(x,y,t)$? How is it related to the coefficients $A_{i,j}$ in (2.5a)? How the multi-scale parameters are coupled?

AUTHOR answer: $A_{i,j}$ is related to $\alpha$ through eq. (2.6c), i.e., the effect of species $i$ on species $j$ (as opposed to how individuals with different trait coordinates effect each other, as is the case at the population level). From a modeling point of view, $\alpha$ defines the ecological interactions in the system, and measures potentially competitive, mutualistic, and trophic interactions, and may in general be either symmetric or unsymmetric. Hence, it may take many different functional forms. For the numerical examples in the paper (which are only meant to illustrate the multi-scale framework, not represent any real eco-evolutionary system) we chose in the first case a symmetric competitive ecology, and in the second case an unsymmetric trophic ecology. More details on $\alpha$ can be found in [2]. However it is outside the scope of the paper to address in detail how $\alpha$ should be chosen in order to model different ecologies.

The strategy for coupling the species and population scales in the multi-scale framework is outlined in section 5 "Speciation", the the main components being the mappings from the species scale to the population scale (reconstruction operator) and vice-versa (compression operator).
\end{enumerate}

\textbf{References:}\\

$[$1$]$ Nordbotten, J. M., Bokma, F., Hermansen, J. S., \& Stenseth, N. C. (2020). "The dynamics of trait variance in multi-species communities". \textit{Royal Society open science}, 7(8), 200321. 

$[$2$]$ Nordbotten, J. M., \& Stenseth, N. C. (2016). "Asymmetric ecological conditions favor Red-Queen type of continued evolution over stasis". \textit{Proceedings of the National Academy of Sciences}, 113(7), 1847-1852.


\end{document} 
