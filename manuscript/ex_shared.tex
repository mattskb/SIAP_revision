% SIAM Shared Information Template
% This is information that is shared between the main document and any
% supplement. If no supplement is required, then this information can
% be included directly in the main document.


% Packages and macros go here
\usepackage{lipsum}
\usepackage{amsfonts}
\usepackage{graphicx}
\usepackage{epstopdf}
\usepackage{algorithmic}
\ifpdf
  \DeclareGraphicsExtensions{.eps,.pdf,.png,.jpg}
\else
  \DeclareGraphicsExtensions{.eps}
\fi

%%%%%%%%%%%%%%%%%%%%%%%%
\usepackage{subcaption}

\newcommand{\vecc}{\mathbf{c}}
\newcommand{\id}{{\mathbb I}}
\newcommand{\tol}{\textnormal{TOL}}
\newcommand{\real}{{\mathbb R}}
\newcommand{\dx}{\textnormal{d}x}
\newcommand{\textn}[1]{\textnormal{#1}}
\newcommand{\pt}{\partial_t}
\newcommand{\dy}{\textnormal{d}y}
\newcommand{\dzeta}{\textnormal{d}\zeta}
\newcommand{\bse}{\begin{subequations}}
\newcommand{\ese}{\end{subequations}}

\DeclareMathOperator*{\argmin}{arg\,min}

\newcommand{\Cbb}{{\mathbb C}}
\newcommand{\Dbb}{{\mathbb D}}

\newcommand{\Ccal}{\mathcal{C}}
\newcommand{\Bcal}{\mathcal{B}}
\newcommand{\Zcal}{\mathcal{Z}}
\newcommand{\Ncal}{\mathcal{N}}
\newcommand{\Lcal}{\mathcal{L}}
\newcommand{\Fcal}{\mathcal{F}}
\newcommand{\Jcal}{\mathcal{J}}
\newcommand{\Rcal}{\mathcal{R}}
\newcommand{\Dcal}{\mathcal{D}}

\newcommand{\sigmabo}{\boldsymbol{\sigma}}
\newcommand{\Sigmabo}{\boldsymbol{\Sigma}}

\newcommand{\vecV}{\mathbf{V}}
\newcommand{\vecN}{\mathbf{N}}
\newcommand{\vecX}{\mathbf{X}}

\newcommand{\dnabla}{\nabla \nabla}
\newcommand{\dnablaii}{\nabla_i \nabla_i}
\newcommand{\dnablajj}{\nabla_j \nabla_j}
\newcommand{\dnablaij}{\nabla_i \nabla_j}

\newcommand{\vertiii}[1]{{\vert\kern-0.25ex\vert\kern-0.25ex\vert #1 \vert\kern-0.25ex\vert\kern-0.25ex\vert}}
\newcommand{\norm}[1]{\lVert#1\rVert}
\newcommand{\bnorm}[1]{\Big\lVert#1\Big\rVert}

\newcommand{\Hdiv}{\mathbf{H}(\divvv,\Omega)}
\newcommand{\divvv}{{\textnormal{div}}}

\newcommand{\etaRK}{\eta_{\textnormal{R},K}}
\newcommand{\etaDF}{\eta_{\textnormal{DF},K}}
\newcommand{\etaphih}{\eta_{\Phi}}
\newcommand{\etagh}{\eta_{g,K}}
\newcommand{\etarh}{\eta_{r,K}}
\newcommand{\etaosch}{\eta_{f}}
\newcommand{\etaphihj}{\eta_{\Phi,i}}

\newcommand{\etaRKj}{\eta_{\textnormal{R},K,i}}
\newcommand{\etaDFj}{\eta_{\textnormal{DF},K,i}}
\newcommand{\etaghj}{\eta_{g,K,i}}
\newcommand{\etarhj}{\eta_{r,K,i}}

\newcommand{\dist}{\textnormal{dist}}
%%%%%%%%%%%%%%%%%%%%%%%%


% Add a serial/Oxford comma by default.
\newcommand{\creflastconjunction}{, and~}

% Used for creating new theorem and remark environments
\newsiamremark{remark}{Remark}
\newsiamremark{hypothesis}{Hypothesis}
\crefname{hypothesis}{Hypothesis}{Hypotheses}
\newsiamthm{claim}{Claim}

%%
\newsiamthm{assum}{Assumption}
\newsiamthm{df}{Definition}
%%

% Sets running headers as well as PDF title and authors
\headers{Modeling the process of speciation using a multi-scale framework including a posteriori error estimates}{M. K. Brun, E. Ahmed, J. M. Nordbotten, and N. C. Stenseth}

% Title. If the supplement option is on, then "Supplementary Material"
% is automatically inserted before the title.
\title{Modeling the process of speciation using a multi-scale framework including a posteriori error estimates\thanks{Submitted to the editors DATE.
\funding{This work was funded 
in part by Norwegian Research Council project no.~263149.}}}

% Authors: full names plus addresses.
\author{Mats K. Brun\thanks{CEES, Dept. of Biosciences, University of Oslo,  
NO-0316 Oslo, Norway 
  (\email{m.k.brun@ibv.uio.no}, \email{n.c.stenseth@mn.uio.no}).}
\and Elyes Ahmed\thanks{SINTEF, NO-0314 Oslo, Norway} 
  (\email{elyes.ahmed@sintef.no}).
\and Jan M. Nordbotten\thanks{Department of Mathematics, University of Bergen,
NO-5020 Bergen, Norway 
  (\email{jan.nordbotten@uib.no}).}
\and Nils Chr. Stenseth\footnotemark[2]}

\usepackage{amsopn}
\DeclareMathOperator{\diag}{diag}


%%% Local Variables: 
%%% mode:latex
%%% TeX-master: "ex_article"
%%% End: 
